\documentclass[12pt, fullpage,letterpaper]{article}

\usepackage[margin=1in]{geometry}
\usepackage{url}
\usepackage{amsmath}
\usepackage{amssymb}
\usepackage{xspace}
\usepackage{graphicx}

\newcommand{\semester}{Spring 2018}
\newcommand{\assignmentId}{3}
\newcommand{\releaseDate}{26 March, 2018}
\newcommand{\dueDate}{4 April, 2018}

\newcommand{\bx}{{\bf x}}
\newcommand{\bw}{{\bf w}}

\title{CS 5350/6350: Machine Learining \semester}
\author{Homework \assignmentId}
\date{Handed out: \releaseDate\\
  Due date: \dueDate}

\begin{document}
\maketitle

\input{emacscomm}
\input{general-instructions}


\section{Simple Conjunction Learner Analysis}
\begin{enumerate}
\item~[50 points] We have derived the PAC guarantee for the elimination algorithm learning simple conjunctions out of $n$ input binary variables. Please use the PAC guarantee (in the lecture slides) to answer the following questions. 
\begin{enumerate}
\item Suppose we have n=100. If we want to ensure  with probability of at least 0.9, the generalization error of the learned hypothesis h is less than 0.1, how many training examples will we at least need? 
\item Suppose n=100. If we want to ensure  with probability of at least 0.99, the generalization error of the learned hypothesis h is less than 0.1, how many training examples will we at least need? 
\item Suppose n=100. If we want to ensure  with probability of at least 0.9, the generalization error of the learned hypothesis h is less than 0.01, how many training examples will we at least need? 
\item Suppose n=100. If we want to ensure  with probability of at least 0.99, the generalization error of the learned hypothesis h is less than 0.01, how many training examples will we at least need? 
\item Suppose n=1000. If we want to ensure  with probability of at least 0.99, the generalization error of the learned hypothesis h is less than 0.01, how many training examples will we at least need? 
\end{enumerate}

\item~[10 points] Show that for $0\le x \le 1$, 
\[
1 - x \le e^{-x}.
\]
\end{enumerate}

\section{Occam's Razor and PAC guarantee for consistent learners}
We have derived the PAC guarantee for consistent learners (namely, the learners can produce a hypothesis which can 100\% accurately classify the training data). The PAC guarantee is described as follows. Let $H$ be the hypothesis space used by our algorithm. Let $C$ be the concept class we want to apply our learning algorithm to search for a target function in $C$. We have shown that,  with probability at least $1-\delta$, a hypothesis $h\in H$ that is consistent with a training set of $m$ examples will have generalization error $\mathrm{err}_D(h) < \epsilon$ if 
\[
m > \frac{1}{\epsilon}\big(\log(|H|) + \log\frac{1}{\delta}\big).
\]

\begin{enumerate}
\item~[20 points] Suppose we have two learning algorithms $L_1$ and $L_2$, which use hypothesis spaces $H_1$ and $H_2$ respectively. We know than $H_1$ is larger than $H_2$, \ie $|H_1| > |H_2|$.
 For each target function in $C$, we assume both algorithms can find a hypothesis consist with the training data. 
 \begin{enumerate}
 \item According to Occam's Razor principle, which learning algorithm's  result hypothesis do you prefer? Why?
 \item How is this principle reflected in our PAC guarantee? Please use the above inequality to explain why we will prefer the corresponding result hypothesis. 
 \end{enumerate}
\end{enumerate}

\section{PAC Learnable Results}
Let us check the PAC learnability for several concept classes. We assume the our learning algorithm can always result a hypothesis consistent with the training data. We assume the hypothesis space $H$ is the same as the concept class $C$. Please determine whether the following concept classes are PAC learnable or not and prove your conclusions. 
\begin{enumerate}
\item~[10 points] General disjunctions out of $n$ binary variables.  
\item~[10 points] $m$-of-$n$ rules (Note that $m$ is a fixed constant).  
\item~[10 points] Simple conjunctions out of $n$ binary variables. 
\item~[10 points] $k$-CNF out of $n$ binary variables. 
\item~[10 points] General boolean functions of $n$ binary variables. 
\end{enumerate}




\end{document}
%%% Local Variables:
%%% mode: latex
%%% TeX-master: t
%%% End:
