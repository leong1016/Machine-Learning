\documentclass[12pt, fullpage,letterpaper]{article}

\usepackage[margin=1in]{geometry}
\usepackage{url}
\usepackage{amsmath}
\usepackage{amssymb}
\usepackage{xspace}
\usepackage{graphicx}
\usepackage{bm}

\newcommand{\semester}{Spring 2018}
\newcommand{\assignmentId}{0}
\newcommand{\releaseDate}{10 January, 2018}
\newcommand{\dueDate}{19 January, 2018}

\newcommand{\bx}{{\bf x}}
\newcommand{\bw}{{\bf w}}

\title{CS 5350/6350: Machine Learining \semester}
\author{Homework \assignmentId\ Solutions\\\\Yulong Liang (u1143816)}

\begin{document}
\maketitle

\input{emacscomm}

\section*{Basic Knowledge}
\label{sec:q1}

\begin{enumerate}
\item~[5 points] 
\textbf{Answer:}
\[
p(A) = 1 - (\frac{1}{2})^{10} = \frac{1023}{1024}
\]

\item~[5 points]
\textbf{Answer:}\\
According to the Venn Diagram,
\[
p(A \cup B) = p(A) + p(B) - p(A \cap B) \le p(A) + p(B)
\]
Only when $p(A \cap B)$ = 0, i.e., event $A$ and event $B$ are disjoint (never happen at the same time), the equality holds.

\item~[10 points]
\textbf{Answer:}\\
According to the proof above,
\begin{align}
p(\cup_{i=1}^n A_i) &= p(\cup_{i=1}^{n-1} A_i \cup A_n)\\
&\le p(\cup_{i=1}^{n-1} A_i) + p(A_n)\\
&\le p(\cup_{i=1}^{n-2} A_i) + p(A_{n-1}) + p(A_n)\\
&\le p(\cup_{i=1}^{n-k} A_i) + p(A_{n-k+1}) + \ldots + p(A_n)\\
&\le p(A_1) + p(A_2) + \ldots + p(A_n)\\
&= \sum_{i=1}^n p(A_i)
\end{align}
Only when all the events $A_1, A_n$ are disjoint (never happen at the same time), the equality holds.

\item~[5 points] 
\textbf{Answer:}\\
According to the axiom of conditional probability,
\setcounter{equation}{0}
\begin{align}
p(A\cap B\cap C) &= p[A\cap (B\cap C)]\\
&= p(A|B\cap C) p(B\cap C)\\
&= p(A|B\cap C) p(B|C) p(C)
\end{align}

\item~[20 points]
        \begin{enumerate}
            \item~[10 points] 
            \begin{enumerate}
            \item
            \textbf{Answer:}
            \begin{table}[h]
            \centering
            \begin{tabular}{ccc}
            \hline\hline
            $X$ & $0$ & $1$ \\ \hline
            $p(X)$ & $3/10$ & $7/10$ \\ \hline\hline
            \end{tabular}
            \hspace{10pt}
            \begin{tabular}{ccc}
            \hline\hline
            $Y$ & $0$ & $1$ \\ \hline
            $p(Y)$ & $4/10$ & $6/10$ \\ \hline\hline
            \end{tabular}
            \end{table}    
            \item
            \textbf{Answer:}
            \begin{table}[h]
            \centering
            \begin{tabular}{ccc}
            \hline\hline
            $X$ & $0$ & $1$ \\ \hline
            $p(X|Y=0)$ & $1/4$ & $3/4$ \\ \hline
            $p(X|Y=1)$ & $1/3$ & $2/3$ \\ \hline\hline
            \end{tabular}
            \hspace{10pt}
            \begin{tabular}{ccc}
            \hline\hline
            $Y$ & $0$ & $1$ \\ \hline
            $p(Y|X=0)$ & $1/3$ & $2/3$ \\ \hline
            $p(Y|X=1)$ & $3/7$ & $4/7$ \\ \hline\hline
            \end{tabular}
            \end{table}
            \item
            \textbf{Answer:}
            \[
            \EE(X) = 1 \cdot p(X=1) + 0 \cdot p(X=0) = 0.7
            \]
            \[
            \EE(Y) = 1 \cdot p(Y=1) + 0 \cdot p(Y=0) = 0.6
            \]
            \[
            \VV(X) = \EE(X^2) - \EE(X)^2 = 0.7 - 0.7^2 = 0.21
            \]
            \[
            \VV(Y) = \EE(Y^2) - \EE(Y)^2 = 0.6 - 0.6^2 = 0.24
            \]            
            \item
            \textbf{Answer:}
            \[
            \EE(Y|X=0) = 1 \cdot p(Y=1|X=0) + 0 \cdot p(Y=0|X=0) = \frac{2}{3}
            \]
            \[
            \EE(Y|X=1) = 1 \cdot p(Y=1|X=1) + 0 \cdot p(Y=0|X=1) = \frac{4}{7}
            \]
            \[
            \VV(Y|X=0) = \frac{2}{3}(1-\frac{2}{3})^2 + \frac{1}{3}(0 - \frac{2}{3})^2 = \frac{2}{9}
            \]
            \[
            \VV(Y|X=1) = \frac{4}{7}(1-\frac{4}{7})^2 + \frac{3}{7}(0 - \frac{4}{7})^2 = \frac{12}{49}
            \]    
            \item
            \textbf{Answer:}
            \[
            Cov(X, Y) = \EE(XY) - \EE(X)\EE(Y) = 0.4 - 0.7 \times 0.6 = -0.02
            \]
            \end{enumerate}
            \item~[5 points]
            \textbf{Answer:}
            \setcounter{equation}{0}
            \begin{gather}
            p(X\cap Y) = 0.4\\
            p(X) \cdot p(Y) = 0.7 \times 0.6 = 0.42\\
            p(X\cap Y) \neq p(X) \cdot p(Y)
            \end{gather}
            Thus, event X and event Y are not independent.
            \item~[5 points] 
            \textbf{Answer:}
            \[
            \EE(Y|X) = 
            \begin{cases}
            \cfrac{2}{3}\ ,\ if\ X=0\\
            \cfrac{4}{7}\ ,\ if\ X=1
            \end{cases}
            \]
            \[
            \EE(X|Y) = 
            \begin{cases}
            0.4\quad,\: if\: Y=0\\
            0.75\quad,\: if\: Y=1
            \end{cases}
            \]
            Thus, they are not constant.
        \end{enumerate}
        
\item~[10 points] 
\textbf{Answer:}\\
According to lognormal distribution,
\begin{enumerate}
\item $\EE(Y)$
\[
\EE(Y) = exp(\mu + \frac{\sigma^2}{2}) = \sqrt{e}
\]
\item $\VV(Y)$
\[
\VV(Y) = exp(2\mu + \sigma^2) \cdot (exp(\sigma^2) - 1) = e \cdot (e - 1) = e^2 - e
\]
\end{enumerate}

\item~[10 points]
\textbf{Answer:}
\begin{enumerate}
\item $\EE(\EE(Y|X)) = \EE(Y)$
\setcounter{equation}{0}
\begin{align}
\EE(\EE(Y|X)) &= \int_{-\infty}^\infty\EE(Y|X=x)f_X(x)dx\\
&= \int_{-\infty}^\infty\int_{-\infty}^\infty yf_{Y|X}(y|x)f_X(x)dydx\\
&= \int_{-\infty}^\infty\int_{-\infty}^\infty y\frac{f_{X, Y}(x, y)}{f_X(x)}f_X(x)dydx\\
&= \int_{-\infty}^\infty\int_{-\infty}^\infty yf_{X, Y}(x, y)dydx\\
&= \int_{-\infty}^\infty yf_Y(y)dy\\
&= \EE(Y)
\end{align}
\item $\VV(Y) = \EE(\VV(Y|X)) + \VV(\EE(Y|X))$
\setcounter{equation}{0}
\begin{align}
&\EE(\VV(Y|X)) + \VV(\EE(Y|X))\\
=& \EE[\EE(Y^2|X) - \EE(Y|X)^2)] + \EE[\EE(Y|X)^2] - \EE[\EE(Y|X)]^2\\
=& \EE[\EE(Y^2|X)] - \EE[\EE(Y|X)^2] + \EE[\EE(Y|X)^2] - \EE[\EE(Y|X)]^2\\
=& \EE(Y^2) - \EE[\EE(Y|X)^2] + \EE[\EE(Y|X)^2] - \EE(Y)^2\\
=& \EE(Y^2) - \EE(Y)^2\\
=& \VV(Y)
\end{align}
\end{enumerate}

\item~[20 points]
\textbf{Answer:}\\
According to the context, the total number of heads c(n) is binomial distrubuted with number of trials equal to n and probability equal to 0.3, i.e., $c(n) \sim B(n, 0.3)$.
\begin{enumerate}
\item $\EE(c(1))$, $\VV(c(1))$
\[
\EE(c(1)) = n \cdot p = 1 \cdot 0.3 = 0.3
\]
\[
\VV(c(1)) = n \cdot p \cdot (1 - p) = 1 \cdot 0.3 \cdot (1 - 0.3) = 0.21
\]
\item $\EE(c(10))$, $\VV(c(10))$
\[
\EE(c(10)) = n \cdot p = 10 \cdot 0.3 = 3
\]
\[
\VV(c(1)) = n \cdot p \cdot (1 - p) = 10 \cdot 0.3 \cdot (1 - 0.3) = 2.1
\]
\item $\EE(c(n))$, $\VV(c(n))$
\[
\EE(c(n)) = n \cdot p = 0.3n
\]
\[
\VV(c(n)) = n \cdot p \cdot (1 - p) = 0.21n
\]
\end{enumerate} 
The expectation and variance of binomial distribution are propotional to the number of trials, namely $\mu = n \cdot p$, $\sigma^2 = n \cdot p \cdot (1-p)$

\item~[10 points]
\textbf{Answer:}
\begin{enumerate}
\item $\nabla f(\x)$
\[
\nabla f(x)
=
\begin{bmatrix}
\cfrac{\partial f(x)}{\partial x_1} \\
\cfrac{\partial f(x)}{\partial x_2} \\
\vdots\\
\cfrac{\partial f(x)}{\partial x_n} \\
\end{bmatrix}
=
\begin{bmatrix}
\cfrac{e^{-\bm{a} \bm{x}} \cdot a_1}{(1 + e^{-\bm{a} \bm{x}})^2} \\
\cfrac{e^{-\bm{a} \bm{x}} \cdot a_2}{(1 + e^{-\bm{a} \bm{x}})^2} \\
\vdots\\
\cfrac{e^{-\bm{a} \bm{x}} \cdot a_n}{(1 + e^{-\bm{a} \bm{x}})^2} \\
\end{bmatrix}
\]
\item $\nabla^2 f(\x)$
\setcounter{equation}{0}
\begin{align}
\nabla^2 f(x)
&=
\begin{bmatrix}
\cfrac{\partial^2 f(x)}{\partial x_1^2} & \cfrac{\partial^2 f(x)}{\partial x_1\partial x_2} & \cdots & \cfrac{\partial^2 f(x)}{\partial x_1\partial x_n}\\
\cfrac{\partial^2 f(x)}{\partial x_2\partial x_1} & \cfrac{\partial^2 f(x)}{\partial x_2^2} & \cdots & \cfrac{\partial^2 f(x)}{\partial x_2\partial x_n} \\
\vdots & \vdots & \ddots & \vdots\\
\cfrac{\partial^2 f(x)}{\partial x_n \partial x_1} & \cfrac{\partial^2 f(x)}{\partial x_n \partial x_2} & \cdots & \cfrac{\partial^2 f(x)}{\partial x_n^2}\\
\end{bmatrix}\\
&=
\begin{bmatrix}
\cfrac{a_1a_1e^{-\bm{a}\bm{x}}(e^{-\bm{a}\bm{x}}-1)}{(1+e^{-\bm{a}\bm{x}})^3} & \cfrac{a_1a_2e^{-\bm{a}\bm{x}}(e^{-\bm{a}\bm{x}}-1)}{(1+e^{-\bm{a}\bm{x}})^3} & \cdots & \cfrac{a_1a_ne^{-\bm{a}\bm{x}}(e^{-\bm{a}\bm{x}}-1)}{(1+e^{-\bm{a}\bm{x}})^3} \\
\cfrac{a_2a_1e^{-\bm{a}\bm{x}}(e^{-\bm{a}\bm{x}}-1)}{(1+e^{-\bm{a}\bm{x}})^3} & \cfrac{a_2a_2e^{-\bm{a}\bm{x}}(e^{-\bm{a}\bm{x}}-1)}{(1+e^{-\bm{a}\bm{x}})^3} & \cdots & \cfrac{a_2a_ne^{-\bm{a}\bm{x}}(e^{-\bm{a}\bm{x}}-1)}{(1+e^{-\bm{a}\bm{x}})^3} \\
\vdots & \vdots & \ddots & \vdots\\
\cfrac{a_na_1e^{-\bm{a}\bm{x}}(e^{-\bm{a}\bm{x}}-1)}{(1+e^{-\bm{a}\bm{x}})^3} & \cfrac{a_na_2e^{-\bm{a}\bm{x}}(e^{-\bm{a}\bm{x}}-1)}{(1+e^{-\bm{a}\bm{x}})^3} & \cdots & \cfrac{a_na_ne^{-\bm{a}\bm{x}}(e^{-\bm{a}\bm{x}}-1)}{(1+e^{-\bm{a}\bm{x}})^3} \\
\end{bmatrix}
\end{align}
\item $\nabla f(\x)$ when $\a = [1,1,1,1,1]^\top$ and $\x = [0,0,0,0,0]^\top$
\[
\nabla f(x)
=
\begin{bmatrix}
\cfrac{e^0 \cdot 1}{(1 + e^0)^2} \\
\cfrac{e^0 \cdot 1}{(1 + e^0)^2} \\
\cfrac{e^0 \cdot 1}{(1 + e^0)^2} \\
\cfrac{e^0 \cdot 1}{(1 + e^0)^2} \\
\cfrac{e^0 \cdot 1}{(1 + e^0)^2} \\
\end{bmatrix}
=
\begin{bmatrix}
1/4 \\
1/4 \\
1/4 \\
1/4 \\
1/4 \\
\end{bmatrix}
\]
\item $\nabla^2 f(\x)$  when $\a = [1,1,1,1,1]^\top$ and $\x = [0,0,0,0,0]^\top$
\setcounter{equation}{0}
\begin{align}
\nabla^2 f(x)
&=
\begin{bmatrix}
\cfrac{a_1a_1e^0(e^0-1)}{(1+e^0)^3} & \cfrac{a_1a_2e^0(e^0-1)}{(1+e^0)^3} & \cdots & \cfrac{a_1a_5e^0(e^0-1)}{(1+e^0)^3}\\
\cfrac{a_2a_1e^0(e^0-1)}{(1+e^0)^3} & \cfrac{a_2a_2e^0(e^0-1)}{(1+e^0)^3} & \cdots & \cfrac{a_2a_5e^0(e^0-1)}{(1+e^0)^3}\\
\cfrac{a_3a_1e^0(e^0-1)}{(1+e^0)^3} & \cfrac{a_3a_2e^0(e^0-1)}{(1+e^0)^3} & \cdots & \cfrac{a_3a_5e^0(e^0-1)}{(1+e^0)^3}\\
\cfrac{a_4a_1e^0(e^0-1)}{(1+e^0)^3} & \cfrac{a_4a_2e^0(e^0-1)}{(1+e^0)^3} & \cdots & \cfrac{a_4a_5e^0(e^0-1)}{(1+e^0)^3}\\
\cfrac{a_5a_1e^0(e^0-1)}{(1+e^0)^3} & \cfrac{a_5a_2e^0(e^0-1)}{(1+e^0)^3} & \cdots & \cfrac{a_5a_5e^0(e^0-1)}{(1+e^0)^3}\\
\end{bmatrix}\\
&=
\begin{bmatrix}
0 & 0 & 0 & 0 & 0 \\
0 & 0 & 0 & 0 & 0 \\
0 & 0 & 0 & 0 & 0 \\
0 & 0 & 0 & 0 & 0 \\
0 & 0 & 0 & 0 & 0 \\
\end{bmatrix}
\end{align}
\end{enumerate}

\item~[5 points]
\textbf{Answer:}\\
Let $f(x)=e^x-(1+x)$, then $f'(x)=e^x-1$, $f''(x)=e^x$. Since $f''(x)=e^x > 0$, $f'(x)$ increase for all $x \in \mathbb{R}$. When $x=0$, $f'(x)=e^0-1=0$. Hence,
\[
\begin{cases}
f'(x)<0 \ ,x \in (-\infty, 0)\\
f'(x)=0 \ ,x = 0\\
f'(x)>0 \ ,x \in (0, +\infty)\\
\end{cases}
\]
From the statement above, $f(x)$ is a convex function which reaches its local minimum at $x=0$, $f_{min}(x)=f(0)=e^0-(1+0)=0$. Thus, $f(x)=e^x-(1+x) \ge 0$, i.e., $1+x \le e^x$.\\\\
Let $g(x)=e^{-x}-(1-x)$, then $g'(x)=-e^{-x}+1$, $g''(x)=e^{-x}$. Since $g''(x)=e^{-x} > 0$, $g'(x)$ increase for all $x \in \mathbb{R}$. When $x=0$, $g'(x)=-e^0-1=0$. Hence,
\[
\begin{cases}
g'(x)<0 \ ,x \in (-\infty, 0)\\
g'(x)=0 \ ,x = 0\\
g'(x)>0 \ ,x \in (0, +\infty)\\
\end{cases}
\]
From the statement above, $g(x)$ is a convex function which reaches its local minimum at $x=0$, $g_{min}(x)=g(0)=e^0-(1-0)=0$. Thus, $g(x)=e^{-x}-(1-x) \ge 0$, i.e., $1-x \le e^{-x}$.
\end{enumerate}

\end{document}